
%%% Local Variables:
%%% mode: latex
%%% TeX-master: t
%%% End:

\documentclass[letterpaper, 12pt]{article}
\usepackage{dialogue}
\usepackage[strict, autopunct]{csquotes}
\usepackage{polyglossia}
\setdefaultlanguage{english}
\usepackage[en-US]{datetime2}

\newcommand{\shorttitle}{???}

\newcommand{\ldjin}{\textit{Long Day's Journey Into Night}}
\newcommand{\oneill}{O'Neill}

\usepackage{titling}
\title{???}
\author{Phoebe Goldman}
\date{\DTMDate{2020-05-13}}

\newcommand{\theprof}{Professor Charlotte Farrell}
\newcommand{\theclass}{History of Drama and Theater 2}

\newcommand{\mlatitle}{\noindent\theauthor{}\hspace*{2cm}
\\\noindent\theprof{}\hspace*{2cm}
\\\noindent\theclass{}\hspace*{2cm}
\\\noindent\thedate{}\hspace*{2cm}
\begin{center}
  \thetitle{}
\end{center}}

\usepackage[margin=1in, headheight=14.5pt]{geometry}

\usepackage{fancyhdr}
\renewcommand{\headrulewidth}{0pt}
\renewcommand{\footrulewidth}{0pt}
\fancyhf{}
\cfoot{\thepage}
\rhead{\shorttitle}
\lhead{\theauthor}
\pagestyle{fancy}

\usepackage{sectsty}
\allsectionsfont{\normalsize\mdseries}

\usepackage{setspace}
\doublespacing{}

\usepackage{hyperref}

\usepackage[
style=authortitle-terse,
backend=biber,
idemtracker=true,
ibidtracker=true,
]{biblatex}

\usepackage{attrib}

\usepackage{ragged2e}
\setlength\RaggedRightParindent{48pt}

% suppresses pp. in citations
\DeclareFieldFormat{postnote}{#1}
\DeclareFieldFormat{multipostnote}{#1}

\addbibresource{bib.bib}

\begin{document}
\RaggedRight
\thispagestyle{plain}

\mlatitle{}

\footnote{As a brief aside, the only copy of \ldjin{} I could get my hands on
  is the Kindle edition, which features neither page nor line numbers. As such,
the best citations I can manage are by the Kindle's \enquote{locations.} My
sincerest apologies.}

When I was young, maybe 12 or 13, I saw a production of Eugene \oneill{}'s
\ldjin{} at the Guthrie Theater. All of the subtext, and much of the text, went
over my head at the time, but still I remember it as one of the best
performances I've ever seen. Looking back now, with a few more years under my
belt and a nose buried in \oneill{}'s text, I see not only the stunning
portrayal of a deeply dysfunctional family who are nonetheless bound together
by their love for one another, but also the construction of an Irish-American
national identity and a serious attempt to cope with the roles of alcohol and
alcohol abuse within that community. \oneill{} wrestles throughout his largely
autobiographical work with competing conceptions of Ireland both as a beautiful
homeland and as a poverty-ridden swamp, with Irish-Americans both as successful
hardworking citizens and as undesirable drunks, with memory both as fond
recollection and as painful haunting, and with alcohol both as a heartening
tonic and as an inescapable disease.

Before all that, though, it is worth nothing a few features of \ldjin{} which
make it unique among plays. Firstly, it is, at least to some extent,
autobiographical. Eugene \oneill{} was the youngest of three children. His
father, James, was born in Ireland, and moved to the United States at the age
of nine, later becoming an actor. Eugene's mother, Mary, was born in America to
immigrant parents. Their oldest child, James \enquote{Jamie} \oneill{} Jr.,
attempted to follow in his father's footsteps, but his acting career was held
back in part by his alcoholism. Their second child, Edmund, was born five years
later, and died of measles at the age of two. Eugene \oneill{} was born 10
years after Jamie, spent much of his early life in boarding school, and
discovered his mother's morphine addiction, which she had developed while
recovering from his birth, at the age of 15. Eugene was diagnosed with
tuberculosis at 24, and began writing plays while recovering in a
sanatorium. Within the course of three years in Eugene's 30s, James Sr. died of
cancer, Mary suffered a fatal stroke, and Jamie drank himself to death. Several
years later, his health declining, \oneill{} wrote a series of three
autobiographical plays, of which \ldjin{} was the second. He instructed his
wife that these plays should never be staged and should be published as books
no earlier than 25 years after his death, but she allowed both publication and
staging a mere three years after cerebral atrophy got the better of him
\parencite[loc. 2826--2862]{timeline}. In \ldjin{}, \oneill{} replaced his
family name with Tyrone, to prevent his readers from interpreting the play as a
strictly accurate autobiographical representation, which it was not, and
swapped his own name with that of his brother Edmund, for reasons which elude
me. As a result of the play's semi-autobiographical nature, we may analyze it
in terms of \oneill{}'s own life; for example, our understanding of Jamie
Tyrone's relationship to alcohol is deepened by the knowledge that it
eventually killed Jamie \oneill{}.

Secondly, as mentioned above, \ldjin{}, despite being formatted as a play, was
never intended to be staged; rather, \oneill{} intended it to be consumed as a
novel. As a result, his stage directions are distinctive both in that they are
incredibly detailed and in that they offer little guidance to actors or
directors, and arguably impose unnecessary restrictions on set designers. For
example, at the beginning of the second act, \oneill{}'s \enquote{stage
  directions} describe, \textquote[{\cite[loc. 709]{ldjin}}]{Edmund sits in the
  armchair at left of table, reading a book. Or rather he is trying to
  concentrate on it but cannot. He seems to be listening for some sound from
  upstairs. His manner is nervously apprehensive and he looks more sickly than
  in the first act.} This will hardly help an actor, who must determine on
their own how best to appear \enquote{nervously apprehensive} or what actions
might portray an inability to focus on a book, nor will it help a makeup
artist, who must decide what features make a person appear \enquote{sickly.} It
does, however, offer great insight into Edmund's physical, mental and moral
state to a reader. A convenient upside is that \oneill{}'s intentions are often
clearer than they might be in a play written for the stage; rather than having
to deduce Edmund's nervous apprehension from directions which said, perhaps,
that he bounced his leg and glanced up from his book, we can simply and
explicitly read his mood off the page.

\printbibliography{}
\end{document}
