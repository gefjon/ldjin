
%%% Local Variables:
%%% mode: latex
%%% TeX-master: t
%%% End:

\documentclass[letterpaper, 12pt]{article}
\usepackage{dialogue}
\usepackage[strict, autopunct]{csquotes}
\usepackage{polyglossia}
\setdefaultlanguage{english}
\usepackage[en-US]{datetime2}

\newcommand{\ldjin}{\textit{Long Day's Journey Into Night}}
\newcommand{\oneill}{O'Neill}

\usepackage{titling}
\title{The ghosts of the four \oneill{}s, and all of Ireland}
\author{Phoebe Goldman}
\date{\DTMDate{2020-05-13}}

\newcommand{\theprof}{Professor Charlotte Farrell}
\newcommand{\theclass}{History of Drama and Theater 2}

\newcommand{\mlatitle}{\noindent\theauthor{}\hspace*{2cm}
\\\noindent\theprof{}\hspace*{2cm}
\\\noindent\theclass{}\hspace*{2cm}
\\\noindent\thedate{}\hspace*{2cm}
\begin{center}
  \thetitle{}\footnote{As a brief aside, the only copy of \ldjin{} I could get my hands on
  is the Kindle edition, which features neither page nor line numbers. As such,
  the best citations I can manage are by the Kindle's approximate
  \enquote{locations.} My sincerest apologies.}
\end{center}}

\usepackage[margin=1.1in, headheight=14.5pt]{geometry}

\usepackage{fancyhdr}
\renewcommand{\headrulewidth}{0pt}
\renewcommand{\footrulewidth}{0pt}
\fancyhf{}
\cfoot{\thepage}
\rhead{\thetitle}
\lhead{\theauthor}
\pagestyle{fancy}

\usepackage{sectsty}
\allsectionsfont{\normalsize\mdseries}

\usepackage{setspace}
\doublespacing{}

\usepackage{hyperref}

\usepackage[
style=authortitle-terse,
backend=biber,
idemtracker=true,
ibidtracker=true,
]{biblatex}

\usepackage{attrib}

\usepackage{ragged2e}
\setlength\RaggedRightParindent{48pt}

% suppresses pp. in citations
\DeclareFieldFormat{postnote}{#1}
\DeclareFieldFormat{multipostnote}{#1}

\addbibresource{bib.bib}

\begin{document}
\RaggedRight
\thispagestyle{plain}

\mlatitle{}

When I was young, maybe 12 or 13, I saw a production of Eugene \oneill{}'s
\ldjin{} at the Guthrie Theater. All of the subtext, and much of the text, went
over my head at the time, but still I remember it as one of the best
performances I've ever seen. Looking back now, with a few more years under my
belt and a nose buried in \oneill{}'s text, I see not only the stunning
portrayal of a deeply dysfunctional family who are nonetheless bound together
by their love for one another, but also the construction of an Irish-American
national identity and a serious attempt to cope with the roles of poverty and
generational trauma within that community.  \oneill{} wrestles throughout his
largely autobiographical work with competing conceptions of Ireland both as a
beautiful homeland and as a poverty-ridden swamp and with the tension between
experiencing the hurtful actions of one's parents and understanding how they
were hurt, in turn, by their parents. \oneill{}'s twin goals (perhaps the first twin
is larger than the second...) were to forgive his now-deceased parents, freeing
himself of their ghosts, and to share that understanding and forgiveness with
the community of Irish-Americans who struggle with the same ghosts.

Before all that, though, it is worth nothing a few features of \ldjin{} which
make it unique among plays. Firstly, it is, at least to some extent,
autobiographical. Eugene \oneill{} was the youngest of three children. His
father, James, was born in Ireland, and moved to the United States at the age
of nine, later becoming an actor. Eugene's mother, Mary, was born in America to
immigrant parents. Their oldest child, James \enquote{Jamie} \oneill{} Jr.,
attempted to follow in his father's footsteps, but his acting career was held
back in part by his alcoholism. Their second child, Edmund, was born five years
later, and died of measles at the age of two. Eugene \oneill{} was born 10
years after Jamie, spent much of his early life in boarding school, and
discovered his mother's morphine addiction, which she had developed while
recovering from his birth, at the age of 15. Eugene was diagnosed with
tuberculosis at 24, and began writing plays while recovering in a
sanatorium. Within the course of three years in Eugene's 30s, James Sr. died of
cancer, Mary suffered a fatal stroke, and Jamie drank himself to death. Several
years later, his health declining, \oneill{} wrote a series of three
autobiographical plays, of which \ldjin{} was the second. He instructed his
wife that these plays should never be staged and should be published as books
no earlier than 25 years after his death, but she allowed both publication and
staging a mere three years after cerebral atrophy got the better of him
\parencite[loc. 2826--2862]{timeline}. In \ldjin{}, \oneill{} replaced his
family name with Tyrone, perhaps to prevent his readers from interpreting the
play as a strictly accurate autobiographical representation, which it was not,
and swapped his own name with that of his brother Edmund, perhaps to avoid
seeming vain, as Edmund Tyrone in the play epitomizes many positive
attributes. As a result of the play's semi-autobiographical nature, we may
analyze it in terms of \oneill{}'s own life; for example, our understanding of
Jamie Tyrone's relationship to alcohol is deepened by the knowledge that it
eventually killed Jamie \oneill{}.

Secondly, as mentioned above, \ldjin{}, despite being formatted as a play, was
never intended to be staged; rather, \oneill{} intended it to be consumed as a
novel. As a result, his stage directions are distinctive both in that they are
incredibly detailed and in that they offer little guidance to actors or
directors, and arguably impose unnecessary restrictions on set designers. For
example, at the beginning of the second act, \oneill{}'s \enquote{stage
  directions} describe, \textquote[{\cite[loc. 709]{ldjin}}]{Edmund sits in the
  armchair at left of table, reading a book. Or rather he is trying to
  concentrate on it but cannot. He seems to be listening for some sound from
  upstairs. His manner is nervously apprehensive and he looks more sickly than
  in the first act.} This will hardly help an actor, who must determine on
their own how best to appear \enquote{nervously apprehensive} or what actions
might portray an inability to focus on a book, nor will it help a makeup
artist, who must decide what features make a person appear \enquote{sickly.} It
does, however, offer great insight into Edmund's physical, mental and moral
state to a reader. A convenient upside is that \oneill{}'s intentions are often
clearer than they might be in a play written for the stage; rather than having
to deduce Edmund's nervous apprehension from directions which said, perhaps,
that he bounced his leg and glanced up from his book, we can simply and
explicitly read his mood off the page.

Every character who appears in \ldjin{}, and a majority of those mentioned are
Irish or Irish-American, and \oneill{} is careful not to let us forget that
fact. The opening stage directions describe Mary by saying that \textquote{her
  face is distinctly Irish in type,} with a nose that is \enquote{long and
  straight,} and a \enquote{mouth wide with full, sensitive lips}. This is
accompanied by the claim that her face \enquote{must once have been extremely
  pretty, and is still striking} \parencite[loc. 127]{ldjin}. Similarly, the
servant, Cathleen, is described at her introduction as
\textquote[{\cite[loc. 709]{ldjin}}]{a buxom Irish peasant, in her early
  twenties, with a red-cheeked comely face, black hair and blue eyes---amiable,
  ignorant, clumsy, and possessed by a dense, well-meaning stupidity}. Though
his description is not entirely flattering, \oneill{} uses Irish physical
features to represent beauty, and throughout the play, Irish behaviors and
speech patterns also stand in for warmth, kindness and happiness. When Tyrone
(aka James Sr.) reassures Mary of her beauty, he delivers his lines
\textquote[{\cite[loc. 368]{ldjin}}]{\textit{with Irish blarney}}. When Mary is
happy, she \textquote[{\cite[loc. 368]{ldjin}}]{\textit{laughs and an Irish
    lilt comes into her voice}}. When she reminisces about her beautiful hair,
she remembers that it was \textquote[{\cite[loc. 368]{ldjin}}]{a rare shade of
  reddish brown and so long it came down below \textins{her} knees}, a
recognizable depiction of a stereotypical Irish woman's hair. (A more modern
and/or less subtle work might have made her ginger, but \oneill{}'s description
is more accurate to the majority population of Ireland.)

It's not just physical or vocal Irish traits which \oneill{} hallmarks,
though. His characters, which is to say, him and his family, pride themselves
on their stubbornness and admire it in others, even when they think they
shouldn't. When Edmund recounts for his father the story of a tenant,
Shaughnessy's, dispute with his neighbor, which Tyrone fears will result in
legal repercussions for himself as the landlord, Tyrone cannot help but
respond:

\begin{dialogue}
  \speak{Tyrone} \direct{Admiringly before he thinks.} The damned old
  scoundrel! By God, you can't beat him! \direct{He laughs---then stops
    abruptly and scowls.} The dirty blackguard! He'll get me in serious trouble
  yet. I hope you told him I'd be mad as hell---

  \speak{Edmund} I told him you'd be tickled to death over the great Irish
  victory, and so you are. Stop faking, Papa.

  \attrib{\cite[loc. 318]{ldjin}}
\end{dialogue}

Shaughnessy's stubbornness and wit in the face of a much more affluent and
powerful man remind the reader of the historical dynamic between Ireland and
England, wherein the Irish have stubbornly stood up for themselves, despite
England's overwhelming social, political, economic and military upper hand, for
centuries. \oneill{} glorifies this same stubbornness during an argument
between Jamie and Tyrone, having Jamie ascribe it to his own analogue, Edmund,
by saying, \textquote[{\cite[loc. 480]{ldjin}}]{I'd like to see anyone
  influence Edmund more than he wants to be. His quietness fools people into
  thinking they can do what they like with him. But he's stubborn as hell
  inside and what he does is what he wants to do, and to hell with anyone
  else!} Later in the same discussion, Tyrone says \textquote{\textit{with a
    touch of pride,} \enquote{Whatever Edmund's done, he's had the guts to go
    off on his own, where he couldn't come whining to me the minute he was
    broke.}} This, too, the image of the self-made man, is a part of the
Irish-American national identity \oneill{} is constructing, in part by showing
Irish-American characters who admire it, and his stubbornness, in his
distinctly Irish-American onstage analogue. It is not only Eugene \oneill{}'s
own analogue who is perceived positively by other characters for having started
at the bottom. Mary tells Jamie off in Act Two, giving him and the reader a
glimpse into her love for Tyrone, by saying,
\textquote[{\cite[loc. 844]{ldjin}}]{Stop sneering at your father! \textelp{}
  You ought to be proud you're his son! He may have his faults. Who hasn't? But
he's worked hard all his life. He made his way up from ignorance and poverty to
the top of his profession! Everyone else admires him\textelp{}}. Here, too, we
see how Irish-Americans admire themselves for working hard to lift themselves
from poor backgrounds to become successful.

Of course, there's a reason why Irish-American immigrants started at the
bottom. Ireland was among the poorest countries in Europe, and a majority of
the immigrants among whom \oneill{} wished to inspire a shared cultural
identity had left during or in the wake of the Great Hunger. For that reason,
\ldjin{} must cope with those immigrants' memories of living
\textquote[{\cite[loc. 455]{ldjin}}]{in a hovel on a bog}. Despite Ireland's
poverty, \oneill{}'s first generation immigrants fiercely defend the homeland
when their children speak ill of it; Tyrone's response to Jamie's
\enquote{hovel on a bog} line is,Whatever Edmund's done, he's had the guts to go off on his own, where
\textquote[{\cite[loc. 455]{ldjin}}]{\textelp{} keep your dirty tongue off
  Ireland, with your sneers about peasants and bogs and hovels!} Later, while
rehashing the same argument, Tyrone similarly bites,
\textquote[{\cite[loc. 1144]{ldjin}}]{And keep your tongue off Ireland! You're
  a fine one to sneer, with the map of it on your face!} In Tyrone's vicious
responses to smears against Ireland, we can see a pride for his homeland which
shapes his identity; he feels that attacks on Ireland are attacks against
himself, and responds in kind by insulting Jamie. This equation of personal
identity with national identity is a defining feature of nationalism, and
while \oneill{} certainly doesn't glorify it in Tyrone, he demonstrates it and
allows his audience to understand its mechanisms.

The shared identity \oneill{} wants to create, though, isn't exclusively, or
even primarily, among first-generation immigrants. Rather, he strives to give
second-generation immigrants like himself and his mother a way to understand
and to describe their experiences watching their parents remember a homeland
they had never seen. In Act Three, Mary teaches Edmund, and through him
\oneill{} and the audience, \textquote[{\cite[loc. 1684]{ldjin}}]{Your father
  is a strange man, Edmund. It took many years before I understood him. You
  must try to understand and forgive him, too, and not feel contempt because
  he's close-fisted}. In many ways, this line summarizes the goal of \ldjin{}:
a memoir or a journal through which \oneill{} can understand, and perhaps
forgive, his family, after contemplating whence they came and how their
experiences shaped them. Though he wished to delay publication, he still
intended to eventually share that same understanding with the wider world, and
his wife Carlotta ensured it happened sooner than \oneill{} had planned. Mary's
attempt to impress understanding on Edmund continues,
\textquote[{\cite[loc. 1697]{ldjin}}]{His father deserted his mother and their
  six children a year or so after they came to America. He told them he had a
  premonition that he would die soon, and he was homesick for Ireland, and
  wanted to go back there to die. So he went and he did die. He must have been
  a peculiar man, too. Your father had to go work in a machine shop when he was
  only ten years old}. I cannot say how much of this is true of James
\oneill{}, but it certainly makes James Tyrone a more sympathetic character to
know that his superstition came from lack of education, his stinginess from
poverty and his anger from abandonment. In showing this, \oneill{} makes the
struggle to understand ones' parents central to Irish-American identity, and
invites those of his audience who share that identity to share in that
struggle.

\section{But how would I do it?}

Staging a Naturalistic or Realistic production of \ldjin{} is challenging
because of its unusual stage directions, though it seems clear to me that it is
intended to be in the style of Realism. That style isn't compelling to me,
though, and so I wouldn't do it. Instead, I would attempt something more in the
style of Brecht's Epic Theater. Since there is no longer a large body of
Irish-Americans with a distinct national identity from other Americans, I think
I'm a good deal too late to make my production focus on that aspect of the
text, despite it having the bulk of my essay. Instead, my performance would be
more explicitly biographical. To that effect, I would change back \oneill{}'s
name changes; the characters in my play would be James, Mary, Jamie and Eugene
\oneill{}.

My production would not attempt to recreate the Tyrones' summer home beneath a
proscenium arch; rather, it would feature a mostly bare thrust stage with a few
small sections of wall, each holding either a door or a window. Between them
would be set a few nondescript chairs around a small table. I'd want to try it
both with and without, but I suspect I'd also end up placing one larger chair,
more like a recliner, somewhere, for the moment when Mary sits on Eugene's lap,
and for Jamie to collapse into in the final act. Beyond this, the only props or
set pieces would be those with which the characters interact during the play,
which to my memory is just three liquor bottles, a pitcher of water and a few
glasses. I'd choose to keep my set simple primarily because I think that trying
to follow \oneill{}'s directions would be both too much work and a futile
endeavor. To communicate what the audience should be imagining onstage, I would
have Eugene read the stage directions. Whenever he did, lights would quickly
dim and a spotlight would illuminate just him. These wouldn't be so dramatic
that it would be hard to see everyone else, just enough to emphasize that
something different was happening. Eugene would, if possible in the moment,
turn to the audience, and read the stage directions aloud, with the
modification that lines describing himself would be changed from third to first
person. For stage directions which describe actions taken without dialogue, the
actors (incl. Eugene) would take them as he spoke, and for those that described
delivery, they would wait for him to finish, the lights would return to normal,
and they would proceed. I think I would end the play by having him read (a
slightly amended version of) its dedication to Charlotta, again changing the
name Tyrone to \oneill{}. Lights would dim, more than prior times, but still
you could see the three other \oneill{}s walk offstage, and then the spotlight
would hit Eugene, and he would say, \textquote[{\cite[loc. 87]{ldjin}}]{Dearest
  \textins{Carlotta}: I give you this play of old sorrow, written in tears and
  blood. A sadly inappropriate gift, it would seem, for a day celebrating
  happiness. But you will understand. I mean it as a tribute to your love and
  tenderness which gave me the faith and love that enabled me to face my dead
  at last and write this play---write it with deep pity and understanding and
  forgiveness for all the four haunted \oneill{}s. \textelp{} \textins{Love,}
  Gene}. Then the lights would cut out for real, Eugene would walk off, the
audience would clap, curtain call, etc.

Again, I'd like to see how it would play before committing to this, but I'd
also try an even further pull from Brecht's book, and use some projectors to
display some added biographical information as the play went on. For example,
during Jamie and Eugene's dialogue in Act Four, when Jamie talks about wanting
to drink himself unconscious, saying,
\textquote[{\cite[loc. 2271]{ldjin}}]{Can't \textins{pass out}, that's
  trouble. Had enough to sink a ship, but can't sink. Well, here's
  hoping. \textit{he drinks}}, a projection would appear reading,
\enquote{Jamie \oneill{} drank himself to death 11 years later in 1922.}

I hope, taken together, that these choices would help to make an audience
realize that \ldjin{} is \oneill{}'s attempt to do the thing Mary told him to
in Act Three, namely, to understand and to forgive. I think that's the most
important feature of this play by a wide margin, but I chose to write the first
part of my essay about Irish-American nationalism instead because of the
assignment that we \enquote{advance an original argument,} since it's hardly a
new claim to say that \oneill{} is using this play to do exactly the thing he
says he's doing in his dedication. On the other hand, though, I didn't worry
too much about anything else in the assignment, so maybe I should've just
written the paper that was obvious to me.

\printbibliography{}
\end{document}
